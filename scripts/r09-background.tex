\documentclass[]{article}
\usepackage{lmodern}
\usepackage{amssymb,amsmath}
\usepackage{ifxetex,ifluatex}
\usepackage{fixltx2e} % provides \textsubscript
\ifnum 0\ifxetex 1\fi\ifluatex 1\fi=0 % if pdftex
  \usepackage[T1]{fontenc}
  \usepackage[utf8]{inputenc}
\else % if luatex or xelatex
  \ifxetex
    \usepackage{mathspec}
  \else
    \usepackage{fontspec}
  \fi
  \defaultfontfeatures{Ligatures=TeX,Scale=MatchLowercase}
\fi
% use upquote if available, for straight quotes in verbatim environments
\IfFileExists{upquote.sty}{\usepackage{upquote}}{}
% use microtype if available
\IfFileExists{microtype.sty}{%
\usepackage{microtype}
\UseMicrotypeSet[protrusion]{basicmath} % disable protrusion for tt fonts
}{}
\usepackage[margin=1in]{geometry}
\usepackage{hyperref}
\hypersetup{unicode=true,
            pdftitle={Vietnam FoF Background},
            pdfauthor={mz---14.12.18},
            pdfborder={0 0 0},
            breaklinks=true}
\urlstyle{same}  % don't use monospace font for urls
\usepackage{graphicx,grffile}
\makeatletter
\def\maxwidth{\ifdim\Gin@nat@width>\linewidth\linewidth\else\Gin@nat@width\fi}
\def\maxheight{\ifdim\Gin@nat@height>\textheight\textheight\else\Gin@nat@height\fi}
\makeatother
% Scale images if necessary, so that they will not overflow the page
% margins by default, and it is still possible to overwrite the defaults
% using explicit options in \includegraphics[width, height, ...]{}
\setkeys{Gin}{width=\maxwidth,height=\maxheight,keepaspectratio}
\IfFileExists{parskip.sty}{%
\usepackage{parskip}
}{% else
\setlength{\parindent}{0pt}
\setlength{\parskip}{6pt plus 2pt minus 1pt}
}
\setlength{\emergencystretch}{3em}  % prevent overfull lines
\providecommand{\tightlist}{%
  \setlength{\itemsep}{0pt}\setlength{\parskip}{0pt}}
\setcounter{secnumdepth}{0}
% Redefines (sub)paragraphs to behave more like sections
\ifx\paragraph\undefined\else
\let\oldparagraph\paragraph
\renewcommand{\paragraph}[1]{\oldparagraph{#1}\mbox{}}
\fi
\ifx\subparagraph\undefined\else
\let\oldsubparagraph\subparagraph
\renewcommand{\subparagraph}[1]{\oldsubparagraph{#1}\mbox{}}
\fi

%%% Use protect on footnotes to avoid problems with footnotes in titles
\let\rmarkdownfootnote\footnote%
\def\footnote{\protect\rmarkdownfootnote}

%%% Change title format to be more compact
\usepackage{titling}

% Create subtitle command for use in maketitle
\newcommand{\subtitle}[1]{
  \posttitle{
    \begin{center}\large#1\end{center}
    }
}

\setlength{\droptitle}{-2em}

  \title{Vietnam FoF Background}
    \pretitle{\vspace{\droptitle}\centering\huge}
  \posttitle{\par}
    \author{mz---14.12.18}
    \preauthor{\centering\large\emph}
  \postauthor{\par}
    \date{}
    \predate{}\postdate{}
  

\begin{document}
\maketitle

{
\setcounter{tocdepth}{3}
\tableofcontents
}
\hypertarget{introduction}{%
\section{Introduction}\label{introduction}}

The goal of the Vietnam Future of Food research project was to explore
the impact of the changing age and gender structure of smallholders on
food production in the context of changing climatic circumstances,
specifically how it affects the farmers' decision-making process with
regard to pest management in conditions of uncertainty. In addition to
gender, age, education, and socio-economic characteristics, other
relevant drivers and opportunities that shape farmers' decision-making
include their past experience, social networks, government mandates,
access to lending, labour availability, migration patterns, knowledge
and training.

This document describes the primary data collection undertaken as part
of this project in the North of Vietnam

\begin{itemize}
\tightlist
\item
  Focus group discussions with 228 smallholder rice farmers that took
  place in August and September of 2015;
\item
  A survey of 413 households actively engaged in farming activities,
  that took place in December of 2016.
\end{itemize}

\hypertarget{focus-groups}{%
\section{Focus Groups}\label{focus-groups}}

\hypertarget{organization}{%
\subsection{Organization}\label{organization}}

\hypertarget{partners-and-roles}{%
\subsubsection{Partners and roles}\label{partners-and-roles}}

The Vietnamese Association of the Elderly (VAE) is one of the largest
\emph{mass organizations} in the country\footnote{The main mass
  organizations are the \emph{Labour Union}, \emph{Women's Union},
  \emph{Farmer's Union}, \emph{Youth Union} and \emph{Veteran Union},
  which are all joined under the umbrella organization \emph{The
  Vietnamese Fatherland Front} These organizations go beyond being civil
  society organizations, as they are funded by the state and given
  certain political responsibilities. The VAE is similar in structure
  and size to the them, albeit it's political power is lesser, but it is
  also considered more independent from the Party.}. Established in 1995
it has over 8.3 million members (out of approx 9.4 million elderly in
the country). Eligibility is at age 60 for men and 55 for women. Most
importantly for our purposes: the organization has chapters at all
levels of the Vietnamese administrative hierarchy: all provinces,
districts and communes, which at the lowest level means over 11,000
branches.

The VAE provided administrative support in organizing the appropriate
visas and research permits to conduct the research as well as the
organizational and logistical support in carrying it out in the field.
Their involvement meant our work was officially sanctioned and local
level authorities allowed the fieldwork to progress without any
interference. Additionally the organizational structure of the VAE meant
that at the local level there were easily contactable officials to help
with the organization and participant selection. It also meant we were
granted free access to facilities within the commune administrative
complex where the focus groups could be held.

At the local level the Vietnamese Farmers' Union (VFU) was our second
partner in organizing the fieldwork. This was necessary due to the
requirement for half the focus group participants to not be
\emph{elderly}, making the VFU the natural choice. As with the VAE, the
VFU is a mass organization, one that additionally has a political
standing in Vietnam, but for our purposes the important aspects are the
same as above: membership with almost complete coverage (over 10,000,000
members) and local branches at all levels of the administrative
hierarchy. The cooperation with the VFU local level branches was
coordinated via the VAE.

Two members of the VAE International Relations Department alternated as
research assistants and focus group facilitators and an interpreter was
engaged to translate during the FGDs. The research assistants were
instrumental in the planning and organization of the fieldwork as well
as in the execution of the focus groups themselves. Both had had
previous experience in running focus groups in similar settings, but
were of course unfamiliar with the novel BBN framework used to conduct
the discussions. The different skill levels of both assistants were one
of the reasons the actual implementation of the groups was adapted (see
below in the Methodology section) and split into two different
\emph{styles}.

\hypertarget{implementation}{%
\subsubsection{Implementation}\label{implementation}}

\hypertarget{focus-group-design}{%
\paragraph{Focus group design}\label{focus-group-design}}

The focus group design was based on three key variables deemed as
relevant to the discussions as well as to obtain reasonable homogeneity
of the groups: (i) age, (ii) gender and (iii) socio-economic status. The
gender breaking variable is non-problematic and was straightforward to
achieve. The age breaking point was based on the standard Vietnamese
retirement age, which is 55 for women and 60 for men. People only become
eligible for VAE membership after this age, although they may still
remain members of the VFU. We additionally asked both organizations to
attempt the following age distributions:

\begin{itemize}
\tightlist
\item
  VAE: one or two participants over 75 years of age
\item
  VFU: half of the participants (four) under 40 years of age.
\end{itemize}

However by and large however these two requirements were not met in
practice see Figure for age distributions.

The final breaking variable was socio-economic status. We label these
groups \emph{deprived} and \emph{not deprived} although in practice it
was difficult to see (as agreed by the Vietnamese research assistants as
well) much difference between the groups\footnote{Vietnamese authorities
  define poor households based on income levels (currently set at
  400,000 VND per person per month - approx £11.50 in rural areas);
  these classifications are used as official measures of eligibility for
  various programmes and are known to organizations such as the VAE and
  VFU. The VAE staff however suggested that there would not be enough
  participants if we split the groups based on this definition alone.
  Even including a second group labelled \emph{near-poor} (currently set
  at 520,000 VND per person per month - approx £15.00) would still be
  difficult in their opinion, so they suggested adding a third group as
  well, described as \emph{difficult life} - meaning their household
  situation was difficult although they did not technically qualify as
  being poor (older people living alone, having to take care of
  grandchildren etc.).}.

A group size of eight was decided upon - on the one hand to avoid too
large a number of participants rendering the discussion too difficult to
manage, and at the same time given the uncertainty of the levels of
attendance we also wanted to avoid too few people turning up. The
discussions were scheduled to last up to 2.5 hours with a short break
for refreshments in between.

This combination of eight groups described above was to be conducted in
each of the four locations (communes). This resulted in 32 scheduled
focus groups based on the described design along with their locations,
however only 31 were conducted, with the final one having to be
cancelled due to too low attendance.

\hypertarget{participant-recruitment}{%
\paragraph{Participant recruitment}\label{participant-recruitment}}

The focus group design as described above was communicated to the local
VAE and VFU chapters about two weeks in advance of the fieldwork
commencing in each of the districts. It was understood that the
organizations would contact and invite participants from within their
membership based on the design requirements, fulfilling the main three
criteria of age, gender and socio-economic categorization, except as
already mentioned, for the age subgroups. It was however unclear, and
impossible to establish, exactly how the sampling was in fact performed,
beyond the description that the participants were selected from a
membership lists.

\hypertarget{consent}{%
\paragraph{Consent}\label{consent}}

All participants in the focus groups gave their consent, and it is my
belief this consent was always informed. Participants were given an
\emph{Information sheet} describing the purpose of the research, the
reasons for the focus groups and their participation and information
about how their information will be used. Consent was obtained from all
focus group participants at the start of each session after the
facilitator gave the introduction.

\hypertarget{locations}{%
\subsection{Locations}\label{locations}}

The selection of the fieldwork sites was based heavily on the inputs and
suggestions of the VAE, with a view of selecting sites where they had
good local knowledge and established relations with the local
communities that did not require additional facilitation. Additional
site criteria were applied to ensure appropriate age and gender
structure, farming practices with a focus on rice farming, communities
with access to lending and agricultural training and support, as well as
being relatively easy to access both geographically and in terms of
permits.

Two districts were chosen in two provinces, and within each district two
communes were selected.

In the Ha Noi province the Hoai Duc district is a district with a
population of 229,000 located immediately to the West of Ha Noi in the
Red River Delta region.

\begin{figure}
\centering
\includegraphics{r09-background_files/figure-latex/unnamed-chunk-2-1.pdf}
\caption{Figure 1: The Hoai Duc district in Ha Noi province}
\end{figure}

Two communes, Yen So and Kim Chung were selected as the locations of the
two sets of focus groups. The area is rapidly urbanising with
agriculture representing around 20\% of the economy.

\begin{figure}
\centering
\includegraphics{r09-background_files/figure-latex/unnamed-chunk-3-1.pdf}
\caption{Figure 2: The Yen So and Kim Chung communes in Hoai Duc
district}
\end{figure}

The Hoang Hoa district in the Thanh Hoa province is a coastal plain
district located in the North Cetral Coast region of Vietnam and has a
slightly larger population of 250,000.

\begin{figure}
\centering
\includegraphics{r09-background_files/figure-latex/unnamed-chunk-4-1.pdf}
\caption{Figure 3: The Hoang Hoa district in Thanh Hoa province}
\end{figure}

The two communes we visited there were Huang Trung and Huang Phu. Over
half the land in the district is agricultural and over 80\% of it is
planted with rice.

\begin{figure}
\centering
\includegraphics{r09-background_files/figure-latex/unnamed-chunk-5-1.pdf}
\caption{Figure 4: The Huang Trung and Huang Phu communes in Hoang Hoa
district}
\end{figure}

\hypertarget{methodology}{%
\subsection{Methodology}\label{methodology}}

The focus groups were designed to use Bayesian Belief Networs (BBNs) as
the conceptual and strucutral framework of the data collection process.
For more details of the focus group methodology can be found see
Založnik et al.~(2018).

The original plan was to build a BBN of all the relevant factors and
their connections in each focus group, however the first set of FGDs
made it clear clear that while the BBN method worked remarkably well in
guiding and structuring the discussion, after a while it also became
quite repetitive, since the network had to be built from scratch each
time, which made it difficult to speed through topics that were
sufficiently covered. Additionally there were concers about both reseach
assistants beign equally skilled at this task. The decision was
therefore made that in the second set of eight groups we would adapt the
discussion framework to \emph{elaborate on an existing BBN} instead of
building a new one each time. These groups were in the same district, so
it was not unreasonable to expect the general context would be similar
enough to the first set of eight.

\hypertarget{bbn-building-focus-groups---from-zaloznik-et-al.-2018}{%
\subsubsection{BBN-Building Focus Groups - from Založnik et
al.~2018}\label{bbn-building-focus-groups---from-zaloznik-et-al.-2018}}

The research assistant was provided with a Focus Group Facilitator's
Guide (see suplementary materials to Založnik et al.~2018).

\begin{quote}
``The aim of the first stage of focus group discussions is to gain
insight into the system as a whole, its most important determinants, and
how these different factors connect causally. The BBN is built with the
active partici- pation of the participants and represents their
collectively negotiated under- standing of their environment. This is
accomplished using a combination of magnetic cards and ribbon
connections.''
\end{quote}

\begin{quote}
``The focus group facilitation guide listed suggested topics for
discussion but made it clear that the list was not exhaustive and that
the moderator should explore any new factors that emerged in the process
(the guide we used in the field is provided as Supplementary Material).
Instead a more abstract set of instructions was given, explaining how
the discussion should be structured and simultaneously mapped onto the
network being built in front of us. This involved two components: new
topics or factors, which are represented by the magnetic cards, and how
they influence each other or causal connections, which are represented
by the ribbons.''
\end{quote}

\begin{quote}
``Each new factor that came up would be written on a card, trying to
ensure it was a simple concept, clearly described, which served as a
discussion point to ensure a common understanding of the factor was
reached. The card then needed to be placed in relation to the existing
factors allowing us to explore and discuss the nature of the
connections. These represent the mechanisms of influence as the
participants understand them and by interrogating the why and the how of
these connections the card would be situated within the network
context.''
\end{quote}

\hypertarget{bbn-elaborating-focus-groups---from-zaloznik-et-al.-2018}{%
\subsubsection{BBN-Elaborating Focus Groups - from Založnik et
al.~2018}\label{bbn-elaborating-focus-groups---from-zaloznik-et-al.-2018}}

\begin{quote}
``The aim of the second round of focus groups is to further elaborate
the framework that was the consolidated outcome of the previous round.
Having established a provisional network structure that is common to all
of the different groups, the flexibility of the network is now secondary
(although it is still amendable) and the moderator can focus more
systematically on questioning how participants' behavioral intentions
are formed.''
\end{quote}

This existing BBN was printed on a large A0 poster and was presented in
each group as the result of discussions with farmers in the previous
commune, while at the same time making it immediately clear that it was
not a \emph{correct} model, and that it was open to debate and
amendments. These were made with markers directly on to the posters,
which were then photographed at the end of each session.

The BBNs used in this set of focus groups further distinguished between
\emph{behaviours} (printed in red) and factors may affect a behaviour or
that are an outcome of a behaviour. Thus each behaviour was connected
with at least one other factor - either an influencing factor (parent
node) or outcome (child node). The framework for the focus group was
grounded in the \emph{Theory of planned behaviour} i.e.~trying to elicit
the (i) attitudes towards a certain behaviour (knowledge and
understanding of the behaviour and it's impacts and utility of it's
outcome), (ii) controls over a certain behaviour (what restricts or
limits a behaviour and how much control does the person have over it)
and (iii) subjective norms in the shape of perceived social norms to
perform a certain behaviour.

\hypertarget{descriptive-statistics}{%
\subsection{Descriptive Statistics}\label{descriptive-statistics}}

\hypertarget{participation-rates}{%
\subsubsection{Participation rates}\label{participation-rates}}

A total of 228 participants took part in 31 focus group discussions. In
each case there were eight participants invited, and most of the time
all of them arrived to take part (17 out of 31), or only one participant
didn't make it (10 out of 31). Only two groups each had two or three
participants missing. This does not include the final 32. group which
got cancelled after only two participants turned up over 45 minutes late
and was therefore cancelled\footnote{The reason for this dramatic
  nonattendance was related to the rain stopping in the afternoon. It
  was peak rice harvest time and the weather had been quite wet, so the
  farmers needed to take advantage of the weather to quickly harvest the
  rice still in the fields and attempt to dry it.}.

Figure 5 summarizes the attendance rate along the four group criteria:
gender, age, socio-economic status and district. Only the chart on the
top left hand side represents relationship where the difference between
groups was significant (at α = 0.05) i.e.~the attendance was higher in
female groups, as well as being higher (p = 0.058) in groups of
participants categorised as deprived (poor, near poor or otherwise de
privileged). Differences in attendance between the different aged groups
(which is the same as the organizational membership) and between the two
districts (which also correlates with the timing of the focus groups)
were not significant.

\begin{figure}
\centering
\includegraphics{r09-background_files/figure-latex/unnamed-chunk-6-1.pdf}
\caption{Figure 5: Number of participants attending FGD by group
characteristics}
\end{figure}

\hypertarget{participant-ages}{%
\subsubsection{Participant Ages}\label{participant-ages}}

Before describing the age structure of the participants we should note
that discrepancies were discovered in their recorded ages. There were
two sources as to the age of the participants. On the one hand the
organizations (VAE or VFU) provided us with a list of the participants
along with their ages, and on the other hand each focus group began with
a short introduction in which the participants stated their ages as
well. While there is some missing data in the second set - it was not
really meant as a data collection exercise, but rather an introduction,
so omissions were not followed up - they are considered the more
reliable source. However upon inspection it transpires that there are
considerable discrepancies between the two sources.\footnote{We can
  consider up to two year discrepancies as expected given the
  idiosyncrasies of how a person's age is commonly expressed in Vietnam.
  Contrary to the standard accounting, a person's age is calculated
  starting from conception (approximately), making them one year old at
  birth, and then increased by one year every new year - not their
  actual birthday. Thus for example a baby born just before the
  Vietnamese new year (Tết - based on the lunar calendar) would already
  be considered 2 years old the next day. It is not clear whether the
  official registers and the self reporting used this colloquial
  Vietnamese counting or the official one and to what extent that is the
  cause of the discrepancies.}

\includegraphics{r09-background_files/figure-latex/unnamed-chunk-7-1.pdf}
Furthermore it becomes quickly obvious that the discrepancies were
dramatically worse in the two communes in Hoằng Hoá district, as is
clear from Figure 6, which plots the ages of the participants from the
two sources against each other, with the green points indicating
complete consistency, blue indicating up to two years difference and
orange the ones that were more than two years out. The data has been
split by district and makes it clear that the issue is location
specific. For the rest of this analysis we use the self-reported data
but in cases these were not recorded (N=23) are they supplemented with
the \emph{official} data.

\begin{figure}
\centering
\includegraphics{r09-background_files/figure-latex/unnamed-chunk-8-1.pdf}
\caption{Figure 7: Age distributions with means in groups from Vietnam
association of the Elderly (top row) and Vietnamese Farmers' Union
(bottom row) for both men (left) and women (right)}
\end{figure}

Figure 7 summarizes the age distribution in the groups divided by age
(i.e.~organization) and gender. Men can become members of the VAE at age
60 and women at age 55 and in both cases only a few participants in the
elderly groups fell under the cut-off point. Although we had asked the
organizers to try to get one to two participants aged over 75, this
proved difficult to accomplish in practice. In the end there were a
total of 7 men over the age of 75 (ages consistent) and only one woman
(self-reported as 75, on official register as 68).

The Farmer's Union on the other hand has no cut off point and people can
stay members even after they reach retirement age, although many do not.
The average age in these groups was around 50 for both genders, although
the spread was much larger for men. This was despite the fact that we
had asked to have half the group under the age of 40. Again this proved
difficult to pull off by the VFU - although it was not clear if this was
due to membership age structure or other reasons. Just under a fifth of
the VFU male participants were under 40, and approximately one tenth of
the women.

\hypertarget{participant-education}{%
\subsubsection{Participant Education}\label{participant-education}}

The education levels of the participants were obtained through
self-reporting on the form they signed at the end of the focus group to
confirm they had received their travel compensation money. Being in the
North of Vietnam, both provinces used to have a 10-year system (somewhat
similar to the USSR one), which was only unified with the Southern
12-year system in 1980. Participants would describe their education as
e.g.~8/10 or 7/12 depending on the system in which they had been
educated.

\begin{figure}
\centering
\includegraphics{r09-background_files/figure-latex/unnamed-chunk-9-1.pdf}
\caption{Figure 8: Recoding of Education variable based on old (10-year)
or new(12-year) system}
\end{figure}

The two systems were recoded into education levels - primary, lower
secondary and high secondary - as they were categorised within each
system (see scheme in Figure 8). This was deemed more appropriate given
that that was the relevant categorisation at the time the participants
were in school, although an alternative categorization based on years
only could also be argued for. Only two participants of the 228 reported
having more than higher secondary school, one a vocational college and
another university degree, both are coded as tertiary education.

Figure 9 summarizes the participants educational attainment by gender
and age group. Both participants with tertiary education were male and
there were a total of three participants that had not completed primary
school, all of them over 70. Overall it there is a clear gender
difference in educational attainment across most age groups as well as a
clear generational trend with younger participants being more likely to
have completed more years of schooling.

\begin{figure}
\centering
\includegraphics{r09-background_files/figure-latex/unnamed-chunk-10-1.pdf}
\caption{Figure 9: Distribution of participants' educational attainment
by age group and gender}
\end{figure}

\hypertarget{size-of-rice-fields-farmed-by-the-participants}{%
\subsubsection{Size of rice fields farmed by the
participants}\label{size-of-rice-fields-farmed-by-the-participants}}

In the introduction phase of the focus groups, the participants were
also asked to report the size of land on which they grow rice. It should
be clear they were not reporting on the area of land they owned - they
may have been renting (part of) the land - nor were other crops
considered. The data therefore refers to the area of land they cultivate
rice on.

The charts in Figure 10 summarise the distribution of land areas between
the different groups. It should be kept in mind that these samples are
not representative, so any patterns should be treated with care when
interpreting. The rice field sizes are grouped into categories for
convenience, with the first three ( under 2 hectares) often considered
smallholder levels of landholdings, however this definition is not
generally accepted and depends on local context, additionally we are
only counting land used for rice farming and farmers may well have
additional land used for other corps that is not counted here.

\begin{figure}
\centering
\includegraphics{r09-background_files/figure-latex/unnamed-chunk-11-1.pdf}
\caption{Figure 10: Distribution of participants' rice field sizes given
gender, organisation membership, deprivation level, commune, education
and age group}
\end{figure}

On the left hand side we first have the difference between the men and
the women. On average women came from households that farmed rice on an
area almost 700m 2 smaller than men (significant at α = 0.05). Similarly
the average landholding of VAE members was smaller by around 640m 2 than
the area farmed by VFU members (also significant at α = 0.05)). The
differences between the so-called deprived and not deprived groups are
minimal (average difference about 150m 2 ) and not significant.

Looking at the differences between the communes only Hoằng Phú was
significantly different from the other three communes, with the average
landholding of the participants around .3 ha larger than in the other
three communes. On the right hand side of the figure we can see the land
are distributions disaggregated by education level and age group. The
largest rice fields are being farmed by the highest educated
participants and the 4-50 year group respectively, however the
differences between the groups are not statistically significant.

\hypertarget{constraints-and-other-issues}{%
\subsection{Constraints and other
issues}\label{constraints-and-other-issues}}

The descriptive analysis in the previous section has several
implications which should be kept in mind during analysis of the
qualitative data as well as during planning the next stage of the
fieldwork.

\hypertarget{organisation-and-logistics}{%
\subsubsection{Organisation and
logistics}\label{organisation-and-logistics}}

The attendance rates confirm the general impression on the ground that
using the VAE and VFU as partners in recruiting the participants and
organizing the focus groups allowed for a high degree of success, at
least in terms of numbers of participants. The fact that only one focus
group got cancelled due to the weather and harvest calendar is quite
remarkable, given that there was plenty of similar situations throughout
the period in question, which did not prevent the participants from
attending anyway. It is not immediately clear to what degree the
financial compensation might have incentivized the participants and to
what degree the official invitation from either of the mass
organizations was interpreted as indicating attendance was compulsory.
My general impression however, corroborated by my research assistants,
was that the authority of the VAE and VFU is not substantial enough to
warrant this level of compliance. Despite being a single party socialist
country, the level of oversight and control enforced by these mass
organizations is relatively weak and the voluntary nature of membership
seems genuinely real.

\hypertarget{participant-sampling}{%
\subsubsection{Participant Sampling}\label{participant-sampling}}

The data analysis shows up some important concerns regarding the way the
participants were selected. The fortuitous fact that we had two sets of
age data allowed us to perform somewhat of an audit on the participant
selection, which turned up some unexpected results. In particular this
refers to the data shown in Figure 6, where it is clear that something
dramatically different was happening in the two Hoằng Hoá communes. One
possible explanation for this inconsistency is that a number of
originally invited participants sent someone else in their place, not
being able to attend themselves. This is partly confirmed by some
inconsistencies in the name lists, although this is more difficult to
confirm, since the names were not deemed important and double record
keeping was not enforced very strictly. Neither the interpreter nor the
research assistants seemed to have noticed that anything like this was
going on at the time.

As far as the qualitative data analysis is concerned, this discrepancy
does not in affect the collected data. The participants - even if they
were not the ones originally invited - fit the requirements of the data
gathering scheme: They were of the correct gender and age, and most
importantly, they were all involved in their households' rice farming
decision making process (the deprived/not deprived aspect is potentially
more problematic and discussed further below).

The issue of the so called deprived and not deprived groups was observed
in the field from the start, namely that there it seemed impossible to
distinguish which group was which without asking a representative of the
organization. The interpreter and research assistants all agreed on
this, and the data analysis seems to confirm there are very few
differences between the groups. In particular one might expect that area
farmed would be significanlty different for the two groups, but that was
not the case. Two points need to be made in this regard. One is that
using the official definitions of poverty, we had originally wanted half
the groups to be composed of participants that classify as such. However
we were informed in our preliminary inquiries that there simply are not
enough people officially classed as poor to fulfil our requirements -
that would have meant 32 participants per commune.

\hypertarget{survey}{%
\section{Survey}\label{survey}}

\hypertarget{organization-1}{%
\subsection{Organization}\label{organization-1}}

\hypertarget{partners}{%
\subsubsection{Partners}\label{partners}}

\hypertarget{implementation-1}{%
\subsubsection{Implementation}\label{implementation-1}}

\hypertarget{locations-1}{%
\subsection{Locations}\label{locations-1}}

\hypertarget{methodology-1}{%
\subsection{Methodology}\label{methodology-1}}

\hypertarget{survey-design-and-sampling}{%
\subsubsection{Survey Design and
Sampling}\label{survey-design-and-sampling}}

\hypertarget{outcomes}{%
\subsection{Outcomes}\label{outcomes}}

\hypertarget{constraints-and-other-issues-1}{%
\subsection{Constraints and other
issues}\label{constraints-and-other-issues-1}}


\end{document}
